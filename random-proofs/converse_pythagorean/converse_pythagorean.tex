\documentclass{article}
\usepackage[utf8]{inputenc}
\usepackage{amssymb}

\title{Proving the Converse of Pythagorean Theorem}
\author{Ethan Xu}
\date{October 2021}

\begin{document}

\maketitle

\section{Introduction}
The Pythagorean Theorem is defined as: if $a^2+b^2=c^2$, then $\theta = 90^{\circ}$. By definition, then the converse of the Pythagorean Theorem is if $\theta = 90^{\circ}$, then $a^2+b^2=c^2$.

\section{Proof}
Let $\bigtriangleup ABC$ be a triangle with sides $a$ ($AB$), $b$ ($BC$), $c$ ($AC$) such that:

\begin{equation}
    a^2+b^2=c^2
\end{equation}

Let $\bigtriangleup DEF$ be a triangle with sides $a$ ($DE$), $b$ ($EF$), $d$ ($DF$). Let $\angle DEF = 90^{\circ}$, such that side $d$ is the hypotenuse. By the Pythagorean Theorem, we have that:

\begin{equation}
    a^2+b^2=d^2
\end{equation}

Now we can substitute $a^2+b^2$ in (1) for $d^2$ from (2), so we have that $c^2=d^2$. Since $c, d > 0$, $c=d$.

By SSS Congruence, we have that $\bigtriangleup ABC \cong \bigtriangleup DEF$, so $\angle ABC = \angle DEF = 90^{\circ}$. Hence, $\bigtriangleup ABC$ contains a right angle. $\hfill\square$
\end{document}
